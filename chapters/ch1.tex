\documentclass[../main.tex]{subfiles}

\graphicspath{{pictures/}{../pictures/}}

\begin{document}

\chapterimage{chapter_head_2.pdf} % Chapter heading image

\chapter{Getting Started}

%----------------------------------------------------------------------------------------
%	Basic Structure
%----------------------------------------------------------------------------------------

\section{Basic Structure of a C Program}\index{Basic Structure}
\subsection{Basics 1}
C is a general purpose language and a rather simple language by today's standards.  This does not mean it's easy to write good C code but rather that the language itself is small. C is also a typed language, meaning you as the programmer have to tell the compiler how to interpret the data flowing through your program.  We do this by assigning types to the data.  We'll talk all about types in due time.  Additionally, C is a compiled language.  This means, we write our source code in a human readable syntax and then a compiler converts that into machine-code ones and zeros our computer can understand.  Lets take a minute or two to discuss how a typical C program is laid out.  Don't worry too much about each and every command.  The point here is to see how to construct our program so that we can later compile and run it. A C source file ends in the ".c" file extension.
\\
\lstinputlisting[caption={\lstname}]{src/01-basics1.c}

\textbf{Line 1}: We import the \textit{stdio.h} \textbf{header}\index{header} file. The \textit{stdio.h} header file is part of the \textit{standard library} which are essentially pre-written functions that are available for our use. This header file in particular informs the compiler about the existence of the \textit{printf} function.\\
\textbf{Line 3}: We declare the \textit{main} function returns an integer.\\
\textbf{Line 4}: The \textit{main} function serves as the entry point into our program and must exist in a C program.  In this example, nothing has been placed between the parentheses which means this function takes no arguments. The function body begins at the opening curly brace "\{" on line 4 and ends at the closing curly brace "\}" on line 7.\\
\textbf{Line 6}: A \textit{return} of \textit{0} indicates successful completion of our program.  A \textit{return} of anything else, indicates there was an error.\\

\subsection{Basics 2}
In the first example, we used the \textit{printf} function which was part the \textit{standard library}.  In the next example, we're going to write our own function.  Lets take a look.
\\
\lstinputlisting[caption={\lstname}]{src/01-basics2.c}

\textbf{Line 3}: Here we declare that \textit{anotherFunc} does not have a return value by using the \textbf{void}\index{void} keyword. Additionally, we use the \textbf{static}\index{function!static} keyword. The K\&R manual \cite{kr_manual} indicates declaring a function \textit{static} limits the scope of the function.  In other words, it doesn't exist outside of this source file and can therefore not be imported by another program. It is good programming practice to limit the scope of your functions and variables to just where they are needed.\\
\textbf{Line 4}: This is a function \textbf{declaration}\index{function!declaration}.  Notice it ends with a semicolon.  It exists here because of line 9.  On line 9, \textit{anotherFunc} is called but at this point, since the compiler is reading the file from top to bottom, it doesn't know about the \textit{anotherFunc} function to include what arguments it requires and what it returns.  We therefore provide a function \textit{declaration} to make the compiler aware of this function, its parameters (arguments), and its return type.  Even if it doesn't know where the function exists, it now knows enough information to know how the function should be called.  We could have written lines 13 thru 16 above our \textit{main} function.  This means that we would not have required the function \textit{declaration}.  However, this example follows the convention of trying to keep the \textit{main} function as close to the top as possible.\\
\textbf{Line 4}: Also included in this line is the keyword \textbf{const}\index{const}\index{const}.  I know we haven't discussed variables yet but essentially the \textit{const} keyword indicates the value of the \textit{msg} variable is "constant" and cannot be changed. This gives the programmer some assurance that the value will not be inadvertently changed inside of \textit{anotherFunc}.\\
\textbf{Line 14}:  Earlier on line 4 we saw a function \textit{declaration}.  This line is a function \textbf{definition}\index{function!definition}.  It must be laid out in the same way as the \textit{declaration}.  This is where the function body is defined and really comes to life.\\

\subsection{Basics 3}
So we've seen how we can \textit{include} code from the \textit{standard library} and we've seen how we can write our own functions.  Now we'll take a look at building our own header\index{header}.  In this way, we can define variables and functions written elsewhere and \textit{include} them in our program.  Why would we want to do this?  One reason may be to keep our files as concise as possible.  In order to do this, we may logically group some of our functions together rather than have them scattered across one very large file.  

While we're at it, we also want to ensure we are following the \textbf{DRY} principle\index{DRY principle}. \textbf{D}on't \textbf{R}epeat \textbf{Y}ourself means that there should ideally be a single function that performs a specific action.  In this way, we are not replicating the same capability multiple times throughout our code.  If we've been adhering to the \textit{DRY} principle and a change is required to that capability, we only need to make the change in one place rather than in multiple places.

A header file assists us in keeping our files concise as well as adhering to the \textit{DRY} principle by allowing us to share functions across our program.  C header files end with the ".h" file extension.

The following example consists of three files:
\begin{enumerate}
	\item \textit{01-helper.h} - Contains various \textit{definitions} and \textit{declarations}.
	\item \textit{01-basics3.c} - Contains our main function.
	\item \textit{01-helper.c} - Contains a single function \textit{definition}.
\end{enumerate}

\lstinputlisting[caption={\lstname}]{src/01-helper.h}

\textbf{01-helper.h - Lines 1, 2, and 13}: These lines form a \textbf{header guard}\index{header!guard}\index{guard|see {header, guard}}.  A \textit{header guard} is a \textit{macro}\index{macro} that ensures that a header file is only pulled in a single time even if \textit{included} into multiple source files.  Otherwise, you may see multiple \textit{definitions} and \textit{declarations} for the same functions and variables strewn across a single program. By convention, the name of the \textit{header guard} will be the all caps version of the filename with an underscore instead of a period.  In my case, \textit{macro} names must start with letters and therefore \textit{01-helper.h} was shortened to \textit{HELPER\_H}.\\
\textbf{01-helper.h - Lines 4 - 10}: This provides the \textit{declaration} and the \textit{definition} for a new type called \textit{mysteryMember\_t}\cite{embedded_c}.\\
\textbf{01-helper.h - Line 12}:  This provides the \textit{declaration} for the \textit{distributeSnacks} function.\\

\lstinputlisting[caption={\lstname}]{src/01-basics3.c}

\textbf{01-basics3.c - Line 2}: On this line, notice the import uses quotation marks around the header file.  This instructs the compiler to look locally for \textit{01-helper.h} rather than in the standard library (typically found in \textit{/usr/include/}).  Without this file, lines 6 and 8 would cause compiler errors.\\
\textbf{01-basics3.c - Line 6}: The type \textit{mysteryMember\_t} \textit{definition} is in the \textit{01-helper.h} header file.  Additionally, the \textit{enum}\index{enum} value assigned to SHAGGY comes from \textit{01-helper.h} as well.\\
\textbf{01-basics3.c - Line 8}: The function \textit{distributeSnacks} \textit{declaration} is also in the \textit{01-helper.h} header file.\\

\lstinputlisting[caption={\lstname}, label={lst:distributeSnacks}]{src/01-helper.c}

\textbf{01-helper.c - Line 1}:  Again, we see the local \textit{01-helper.h} header file being imported.  Without this, \textit{01-helper.c} would not know about the \textit{mysteryMember\_t} type that is on line 4  or the enum values on lines 8 and 11.\\
\textbf{01-helper.c - Line 4}:  This is the function \textit{definition} for distributeSnacks that defines what this function does.\\

We have three files that will all get compiled together into a single executable.  The \textit{01-helper.h} header file is what makes all of this possible.  In the next section, we'll talk about what the compiler is, what it does, and how to use it.

%----------------------------------------------------------------------------------------
%	Compilation
%----------------------------------------------------------------------------------------

\section{Compilation}

At a very high level, \textbf{compiling}\index{compile} a program is the act that takes raw source code into machine code that can be executed.  A common compiler, and the one I'll be using in this book, is the \textit{GCC}\index{gcc} compiler.  It is available on nearly every UNIX based system.  You can invoke it directly and in the following examples we will do so.  Later on we'll see how we can leverage the \textit{make} utility and \textit{Makefiles} to make this easier and repeatable.

There are distinct steps that a compiler goes through in order to produce an executable.  These steps generally fall into four phases: \textit{preprocessing}, \textit{compiling}, \textit{assembling}, and \textit{linking}\cite{c_nutshell}.

\paragraph{Preprocessing}\index{preprocessor}

During this phase, the source code is cleaned up with consistent new lines and line continuations (the slash) are removed.  Comments are also removed as they a not needed.  The source code is further modified to expand out \textit{macros} and \textit{preprocessor directives}.  For example, the \textit{header guard} we saw earlier must be processed accordinly.  The \textit{preprocessor} also pulls in any header files specifed with the \textit{\#include} directive.  The output from this phase is just modified source code.  This source code can be output by passing the \textit{-E} option to \textit{gcc}.

\paragraph{Compiling}\index{compiler}

At this point, the source code is ready to be translated into something closer to what the machine can understand.  In this phase the source code is translated into assembly code.  Assembly code is still human readable but is architecture specific.  This assembly code can be output by \textit{gcc} by passing the \textit{-S} option.

\paragraph{Assembling}\index{assembler}

Since assembly code is architecture specific, the assembly code is now passed to an \textit{assembler}. The results of this phase is an object file ending with a ".o" file extension.  This object file contains the machine code of the program that was just compiled minus any external dependencies that were included.

\paragraph{Linking}\index{linker}

Now that all source files have been compiled into object files, it is the job of the \textit{linker} to link them together.  For example, because we used the \textit{printf} function from the \textit{standard library}, the \textit{libc} static library will be automatically linked in.  However, for other libraries, this process is not automatic.  For example, as we'll see later in this book, when we begin using \textit{pthreads}, we have to tell the linker to link to the \textit{libpthread} library by passing \textit{gcc} the \textit{-pthread} option or by adding \textit{-pthread} to the \textit{LDLIBS} environment variable.

\subsection{gcc}


\begin{lstlisting}[language=bash, numbers=none]
$ ls
01-basics1.c

$ gcc 01-basics1.c

$ ls -laF
...
-rw-rw-r-- 1 ron ron    88 Jun 26 10:30 01-basics1.c
-rwxrwxr-x 1 ron ron 16696 Jun 26 12:22 a.out*

$ ./a.out
I can output to the screen!
\end{lstlisting}

On our Linux terminal, we run \textit{gcc} giving it our source code and it generates an executable called \textit{a.out}.  When we run it, we get the output "I can output to the screen!" to out terminal. \textit{a.out} is the default name given to the executable because we didn't tell the gcc compiler what name we wanted.  To do this, we use the \textit{-o} option.\\

\begin{lstlisting}[language=bash, numbers=none]
$ gcc -o 01-basics1 01-basics1.c

$ ls -laF
...
-rwxrwxr-x 1 ron ron 16696 Jun 26 12:34 01-basics1*
-rw-rw-r-- 1 ron ron    88 Jun 26 10:30 01-basics1.c
 
$ ./01-basics1 
I can output to the screen!
\end{lstlisting}

Up until now we've seen a single file being compiled with \textit{gcc} but in our Basic3 example, there were three files.  To compile that, we just need to specify the additional files.\\

\begin{lstlisting}[language=bash, numbers=none]
$ gcc -o 01-basics3 01-basics3.c 01-helper.h 01-helper.c 

$ ls -laF
...
-rwxrwxr-x 1 ron ron 16776 Jun 26 13:08 01-basics3*
-rw-rw-r-- 1 ron ron   166 Jun 26 10:30 01-basics3.c
-rw-rw-r-- 1 ron ron   221 Jun 26 10:30 01-helper.c
-rw-rw-r-- 1 ron ron   160 Jun 26 10:30 01-helper.h
...

$ ./01-basics3
I get 20 scooby snacks!
\end{lstlisting}

So now we can compile our source code into an executable with the name of our choosing but still we're missing out on a bunch of checking we could have the compiler perform.\\

\begin{lstlisting}[language=bash, numbers=none]
$ gcc -std=c11 -Wall -Wextra -Wpedantic -Wwrite-strings \
> -Wfloat-equal -Waggregate-return -Winline -Wvla \
> -o 01-basics1 01-basics1.c

$ ls -laF
...
-rwxrwxr-x 1 ron ron 16696 Jun 26 12:40 01-basics1*
-rw-rw-r-- 1 ron ron    88 Jun 26 10:30 01-basics1.c

$ ./01-basics1 
I can output to the screen!
\end{lstlisting}

In this example, we compiled with the following flags:
\begin{itemize}
	\item \textit{-std=c11} - The 2011 ISO C++ standard plus amendments.
	\item \textit{-Wall} - This enables all the warnings about constructions that some users consider questionable, and that are easy to avoid (or modify to prevent the warning), even in conjunction with macros.
	\item \textit{-Wextra} - This enables some extra warning flags that are not enabled by -Wall.
	\item \textit{-Wpedantic} - Issue all the warnings demanded by strict ISO C and ISO C++; reject all programs that use forbidden extensions, and some other programs that do not follow ISO C and ISO C++.  For ISO C, follows the version of the ISO C standard specified by any -std option used.
	\item \textit{-Wwrite-strings} - When compiling C, give string constants the type "const char[length]" so that copying the address of one into a non-"const" "char *" pointer produces a warning.  These warnings help you find at compile time code that can try to write into a string constant, but only if you have been very careful about using "const" in declarations and prototypes.
	\item \textit{-Wfloat-equal} - Warn if floating-point values are used in equality comparisons.
	\item \textit{-Waggregate-return} - Warn if any functions that return structures or unions are defined or called.
	\item \textit{-Winline} - Warn if a function that is declared as inline cannot be inlined.
	\item \textit{-Wvla} - Warn if a variable-length array is used in the code.
\end{itemize}

The above options or flags are a small sample of what can be chosen.  However, they represent common issues with compiling C code and therefore are the flags I used in compiling the examples for this book.  The descriptions were pulled from referencing the \textit{gcc} manual page by running "\texttt{man gcc}".

Additional information can be found on the GNU website at \href{https://gcc.gnu.org/onlinedocs/gcc/Option-Summary.html\#Option-Summary}{https://gcc.gnu.org/onlinedocs/gcc/Option-Summary.html\#Option-Summary}.

%----------------------------------------------------------------------------------------
%	Make
%----------------------------------------------------------------------------------------

\subsection{Make}

When you're compiling often or compiling a large number of files, typing in the \textit{gcc} command manually can be cumbersome and error prone.  Luckily GNU has a utility called \textit{make}\index{make}.  It makes compiling much easier.\\

\begin{lstlisting}[language=bash, numbers=none]
$ make 01-basics3
gcc     01-basics3.c   -o 01-basics3

$ ls -laF
...
-rwxrwxr-x 1 ron ron 16696 Jun 26 14:28 01-basics1*
-rw-rw-r-- 1 ron ron    88 Jun 26 10:30 01-basics1.c
...

$ ./01-basics1 
I can output to the screen!
\end{lstlisting}

This slightly shortens how much we need to type but to see the real power behind the \textit{make} utility, you need to create a file called \textit{Makefile}\index{Makefile}.  Inside of that file, you can specify all sorts of things that will happen when you type \textit{make} in the same directory as the \textit{Makefile}.  \textit{Makefiles} are highly customizable.  Below is a part of the example I use for compiling the code for this book./\\

\lstinputlisting[caption={\lstname}, language=make]{src/Makefile}

\textbf{Line 3}: This line indicates we want to use the \textit{gcc} compiler.\\
\textbf{Lines 4 and 5}: These lines set our compile time flags.  Every time we compile, these are the flags that will be used to reduce mistakes on forgetting to add one or mistyping another.\\
\textbf{Lines 8 and 11}: These wildcards identify all is all of our ".c" files and \textit{BINS} is the name of the resulting binaries.\\
\textbf{Line 14}: Identifies the targets below that do not refer to actual filenames.\\
\textbf{all}: By convention, the first target is \textit{all}.  Since its the first, it will build each target specified.  By listing my three programs there, the matching ".c" files will be build by \textit{make} as you see below.\\
\textbf{01-basics3}: Since \textit{01-basics3} required two additional files (prerequisites) to build, those files are specified.  This takes the form of:
\begin{verbatim}
target: prerequisites
<TAB> recipe
\end{verbatim}

\textbf{clean}: Here we define a target called \textit{clean} and specify the recipe is to run \textit{rm -rvf} (remove verbosely) against all local object files (files ending in ".o") as well as the compiled binaries.\\
\begin{lstlisting}[language=bash, numbers=none]
$ make
gcc -std=c11 -Wall -Wextra -Wpedantic -Wwrite-strings \
-Wfloat-equal -Waggregate-return -Winline -Wvla \
01-basics1.c   -o 01-basics1

gcc -std=c11 -Wall -Wextra -Wpedantic -Wwrite-strings \
-Wfloat-equal -Waggregate-return -Winline -Wvla \
01-basics2.c   -o 01-basics2

gcc -std=c11 -Wall -Wextra -Wpedantic -Wwrite-strings \
-Wfloat-equal -Waggregate-return -Winline -Wvla \
01-basics3.c 01-helper.c 01-helper.h   -o 01-basics3

$ ls -laF
...
-rwxrwxr-x 1 ron ron 16696 Jun 26 15:08 01-basics1*
-rw-rw-r-- 1 ron ron    88 Jun 26 10:30 01-basics1.c
-rwxrwxr-x 1 ron ron 16784 Jun 26 15:08 01-basics2*
-rw-rw-r-- 1 ron ron   224 Jun 26 10:30 01-basics2.c
-rwxrwxr-x 1 ron ron 16776 Jun 26 15:08 01-basics3*
-rw-rw-r-- 1 ron ron   166 Jun 26 10:30 01-basics3.c
-rw-rw-r-- 1 ron ron   221 Jun 26 10:30 01-helper.c
-rw-rw-r-- 1 ron ron   160 Jun 26 10:30 01-helper.h
-rw-rw-r-- 1 ron ron   595 Jun 26 14:59 Makefile

$ make
make: Nothing to be done for 'all'.

$ make clean
rm -rvf *.o 01-basics1 01-basics2 01-helper 01-basics3
removed '01-basics1'
removed '01-basics2'
removed '01-basics3'
\end{lstlisting}

You can see by running \textit{make} a single time, it compiles all of our files with the approriate compile time flags thereby eliminating a lot of typing and the possibility of missing a flag.  Additionally, if we run \textit{make} again, it already knows the binaries exist, and the source files have not changed, so there is no reason to recompile.  Lastly we run \textit{make clean} which automatically removes our binaries.  There is quite a bit of other functionality that you can build into your \textit{Makefiles}.  I invite you to check out: \href{https://www.gnu.org/software/make/manual/make.html}{https://www.gnu.org/software/make/manual/make.html}

\end{document}
