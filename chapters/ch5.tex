\documentclass[../main.tex]{subfiles}

\graphicspath{{pictures/}{../pictures/}}

\chapterimage{chapter_head_5.pdf} % Chapter heading image

\begin{document}
	
	%----------------------------------------------------------------------------------------
	%	CLI
	%----------------------------------------------------------------------------------------
	\chapter{Structures}\index{struct}
	
	We've now seen enough of the mechanics to talk about building a \textbf{structure}; better known as a \textbf{struct}. These allow us to build more complex or custom data types.  Unlike arrays, \textit{structs} allow for multiple different data types to be placed inside of a single object.  Often times, instead of passing around the struct itself, you'll see pointers to \textit{structs} being passed around instead.  This reduces the amount of data that has to be added to the stack or passed between functions.  
	
	In the following example, we'll use multiple \textit{structs} to represent a player capable of carrying items.\\
	
	\lstinputlisting[caption={\lstname}, label={lst:playerStructH}]{src/05-player.h}	
	
	There are three different \textit{structs} here: \textit{player} (line 17), \textit{inventory} (line 11), and \textit{item} (line 7).  Rather than have a single \textit{struct}, they are broken up by purpose to keep our player organized as we continue to build it out. Notice that a \textit{player struct} contains a single \textit{inventory struct}.  Also notice that an \textit{inventory struct} contains a pointer to an \textit{item struct}.  As we'll see in the next file, we'll use this pointer to actually allocate for an array of \textit{item structs}.  By doing this, a single player can have one inventory that contains multiple items.
	
	As should be easy to see, a \textit{struct} \textit{definition}\index{definition} begins with the \textit{struct} statement followed by the name of the \textit{struct}.  Inside the brackets we specify each member of the struct.
	
	On line 17, we can also see we're using the keyword \textit{typedef} and then on line 20 we're giving it the name \textit{player\_t}.  This allows us to refer to a \textit{player struct} as \textit{player\_t} rather than \textit{struct player}. Its customary to provide a \textit{typedef} when the user will be interacting with it. \\
	
	\lstinputlisting[caption={\lstname}, label={lst:playerStructC}]{src/05-player.c}
	
	I won't go over all of the above code.  I'll leave that as an exercise for the reader.  However, notice on line 14 that we use \textit{calloc} to allocate for a \textit{player struct}.  I'm using \textit{calloc} because all values will start at zero and all pointers will start as NULL.  This is convenient when initially setting up a \textit{struct} where you may not be setting initial values for each member.  On lines 30 and 31 we also use \textit{calloc} to allocate for our array of pointers to \textit{item structs}.  By doing this it sets the initial \textit{char} array for \textit{item struct names} to \textit{NULL}.  The number of pointers in this array is based on the value specified in \textit{maxItems}.
	
	Notice on line 29 that in order to access the \textit{inv} member from our \textit{player} we use "->".  \textit{Player} is actually a pointer to a \textit{player struct} and the "->" operator first dereferences the pointer before accessing the \textit{inv} member.  However, the \textit{inv} member is not a pointer and so it's members can be directly accessed via the "." operator.  Also notice that I'm checking for \textit{NULL} quite a bit throughout this code.  Anytime you are working with pointers and dynamically created memory, you have to be careful not to reach into a \textit{NULL} pointer and cause a SEGMENTATION fault.
	
	The last thing I would like to point out is the \textit{destroyPlayer} function.  While this function is a bit overkill, I wanted to point out a few things.  First off, because we allocated for most of the memory used, we must run \textit{free} on each piece that we allocated.  Additionally, the order is nearly in the opposite order in which it was first allocated.  By taking a bottom to top approach we ensure that we don't lose access to memory that we haven't gotten a chance to \textit{free} yet.  So, in this example, we first free each \textit{name} of an item, then we \textit{free} the array of pointers to items, we then \textit{free} the name of the player, and then finally the \textit{player} itself.  
	
	Also worth pointing out is that I set each pointer to NULL after I \textit{free} it.  I do this only to point out that the use of \textit{free} doesn't actually change the value of the pointer.  If you were to re-use the pointer later in a program and test for \textit{NULL} before attempting to access it, you'll get a false sense of security as the underlying memory no longer belongs to you and will result in a SEGMENTATION fault.  If there is the slightest chance the pointer will get used again, set it to \textit{NULL} after you \textit{free} it.  Again, this isn't entirely required in this scenario and is only here for demonstration purposes.  The one line that is arguably required is line 113.  Because the \textit{destroyPlayer} is taking the address of a pointer to a \textit{player struct}, line 113 will ensure that the original pointer ends up as \textit{NULL} in the calling function.\\
	
	\lstinputlisting[caption={\lstname}]{src/05-struct1.c}
	
	There isn't much here to discuss.  The \textit{cratePlayer} function returns a pointer to a newly created \textit{player struct}.  Here we use the \textit{typedef} name \textit{player\_t} rather than \textit{struct player} because its more convenient. When we call \textit{destroyPlayer} we pass in the address of our our \textit{player1} pointer.  
	
	Let run our program and make sure all of the memory is properly released.
	
	\begin{verbatim}
	$ valgrind ./05-struct1 
	==65023== Memcheck, a memory error detector
	==65023== Copyright (C) 2002-2017, and GNU GPL'd, by Julian Seward et al.
	==65023== Using Valgrind-3.15.0 and LibVEX; rerun with -h for copyright info
	==65023== Command: ./05-struct1
	==65023== 
	Enter a name for the player: Ron
	Enter an item name: Sword 
	Successfully added.
	Enter an item name: Dental Floss
	Successfully added.
	Player: Ron
	    Item 1: Sword
	    Item 2: Dental Floss
	Successfully destroyed.
	==65023== 
	==65023== HEAP SUMMARY:
	==65023==     in use at exit: 0 bytes in 0 blocks
	==65023==   total heap usage: 7 allocs, 7 frees, 2,480 bytes allocated
	==65023== 
	==65023== All heap blocks were freed -- no leaks are possible
	==65023== 
	==65023== For lists of detected and suppressed errors, rerun with: -s
	==65023== ERROR SUMMARY: 0 errors from 0 contexts (suppressed: 0 from 0)
	\end{verbatim}  
		 
	As we can see, there is no memory in use on exit and there are no errors.  
	
\end{document}